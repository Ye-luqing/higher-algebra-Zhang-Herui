\documentclass[twoside,11pt]{article} 
\usepackage{amsmath,amsfonts,bm}
\usepackage{hyperref}
\usepackage{amsthm} 
\usepackage{amssymb}
\usepackage{framed,mdframed}
\usepackage{graphicx,color} 
\usepackage{mathrsfs,xcolor} 
\usepackage[all]{xy}
\usepackage{fancybox} 
% \usepackage{CJKutf8}
\usepackage{xeCJK}
\newtheorem{theorem}{定理}
\newtheorem{lemma}{引理}
\newtheorem{corollary}{推论}
\newtheorem*{exercise}{习题}
\newtheorem*{example}{例}
\newtheorem{remark}{注}
\setCJKmainfont[BoldFont=Adobe Heiti Std R]{Adobe Song Std L}
% \usepackage{latexdef}
\def\ZZ{\mathbb{Z}} \topmargin -0.40in \oddsidemargin 0.08in
\evensidemargin 0.08in \marginparwidth 0.00in \marginparsep 0.00in
\textwidth 16cm \textheight 24cm \newcommand{\D}{\displaystyle}
\newcommand{\ds}{\displaystyle} \renewcommand{\ni}{\noindent}
\newcommand{\pa}{\partial} \newcommand{\Om}{\Omega}
\newcommand{\om}{\omega} \newcommand{\sik}{\sum_{i=1}^k}
\newcommand{\vov}{\Vert\omega\Vert} \newcommand{\Umy}{U_{\mu_i,y^i}}
\newcommand{\lamns}{\lambda_n^{^{\scriptstyle\sigma}}}
\newcommand{\chiomn}{\chi_{_{\Omega_n}}}
\newcommand{\ullim}{\underline{\lim}} \newcommand{\bsy}{\boldsymbol}
\newcommand{\mvb}{\mathversion{bold}} \newcommand{\la}{\lambda}
\newcommand{\La}{\Lambda} \newcommand{\va}{\varepsilon}
\newcommand{\be}{\beta} \newcommand{\al}{\alpha}
\newcommand{\dis}{\displaystyle} \newcommand{\R}{{\mathbb R}}
\newcommand{\N}{{\mathbb N}} \newcommand{\cF}{{\mathcal F}}
\newcommand{\gB}{{\mathfrak B}} \newcommand{\eps}{\epsilon}
\renewcommand\refname{参考文献} \def \qed {\hfill \vrule height6pt
  width 6pt depth 0pt} \topmargin -0.40in \oddsidemargin 0.08in
\evensidemargin 0.08in \marginparwidth0.00in \marginparsep 0.00in
\textwidth 15.5cm \textheight 24cm \pagestyle{myheadings}
\markboth{\rm \centerline{}} {\rm \centerline{}}
\begin{document}
\title{\huge{\bf{习题2.3.3}}} \author{\small{叶卢
    庆\footnote{叶卢庆(1992---),男,杭州师范大学理学院数学与应用数学专业
      本科在读,E-mail:h5411167@gmail.com}}\\{\small{杭州师范大学理学院,浙
      江~杭州~310036}}} \date{}
\maketitle

% ----------------------------------------------------------------------------------------
% ABSTRACT AND KEYWORDS
% ----------------------------------------------------------------------------------------





\vspace{30pt} % Some vertical space between the abstract and first section
\begin{exercise}[2.3.3]
令 $f(x),g(x)$ 是 $\mathbf{F}(x)$ 的多项式,而 $a,b,c,d$ 是
$\mathbf{F}$ 中的数,并且
$$
\begin{vmatrix}
  a&b\\
c&d\\
\end{vmatrix}\neq 0,
$$
证明,
$$
(af(x)+bg(x),cf(x)+dg(x))=(f(x),g(x))
$$  
\end{exercise}
\begin{remark}
  我们先来看与之对应的数论中的相应结论.$a,b,c,d$ 仍然满足题设条件,但是
  $p,q\in \mathbf{Z}$,我们来证明
$$
(ap+bq,cp+dq)=(p,q).
$$
很可惜这个命题是不成立的.比如,$(1,1)=1$,但是 $(1+2,3+9)=3$.由以下论证
可以看出,对于数论中的对应结论,我们必须要求 $ad-bc=\pm 1$.
\end{remark}
\begin{proof}[\bf{证明}]
令 
$$
\begin{cases}
  af(x)+bg(x)=p(x),\\
cf(x)+dg(x)=q(x).
\end{cases}
$$
由于行列式不为0,因此可得
$$
\begin{cases}
  f(x)=\frac{\begin{vmatrix}
      p(x)&b\\
q(x)&d
    \end{vmatrix}}{\begin{vmatrix}
      a&b\\
c&d
    \end{vmatrix}},\\
g(x)=\frac{\begin{vmatrix}
    a&p(x)\\
c&q(x)
  \end{vmatrix}}{\begin{vmatrix}
    a&b\\
c&d
  \end{vmatrix}}.
\end{cases}
$$
这样子,就易得命题成立(为什么?).
\end{proof}
% ----------------------------------------------------------------------------------------
% ESSAY BODY
% ----------------------------------------------------------------------------------------

% BIBLIOGRAPHY
% ----------------------------------------------------------------------------------------
% 
% ----------------------------------------------------------------------------------------
\end{document}








