\documentclass[twoside,11pt]{article} 
\usepackage{amsmath,amsfonts,bm}
\usepackage{hyperref}
\usepackage{amsthm} 
\usepackage{amssymb}
\usepackage{framed,mdframed}
\usepackage{graphicx,color} 
\usepackage{mathrsfs,xcolor} 
\usepackage[all]{xy}
\usepackage{fancybox} 
% \usepackage{CJKutf8}
\usepackage{xeCJK}
\newtheorem{theorem}{定理}
\newtheorem{lemma}{引理}
\newtheorem{corollary}{推论}
\newtheorem*{exercise}{习题}
\newtheorem*{example}{例}
\setCJKmainfont[BoldFont=Adobe Heiti Std R]{Adobe Song Std L}
% \usepackage{latexdef}
\def\ZZ{\mathbb{Z}} \topmargin -0.40in \oddsidemargin 0.08in
\evensidemargin 0.08in \marginparwidth 0.00in \marginparsep 0.00in
\textwidth 16cm \textheight 24cm \newcommand{\D}{\displaystyle}
\newcommand{\ds}{\displaystyle} \renewcommand{\ni}{\noindent}
\newcommand{\pa}{\partial} \newcommand{\Om}{\Omega}
\newcommand{\om}{\omega} \newcommand{\sik}{\sum_{i=1}^k}
\newcommand{\vov}{\Vert\omega\Vert} \newcommand{\Umy}{U_{\mu_i,y^i}}
\newcommand{\lamns}{\lambda_n^{^{\scriptstyle\sigma}}}
\newcommand{\chiomn}{\chi_{_{\Omega_n}}}
\newcommand{\ullim}{\underline{\lim}} \newcommand{\bsy}{\boldsymbol}
\newcommand{\mvb}{\mathversion{bold}} \newcommand{\la}{\lambda}
\newcommand{\La}{\Lambda} \newcommand{\va}{\varepsilon}
\newcommand{\be}{\beta} \newcommand{\al}{\alpha}
\newcommand{\dis}{\displaystyle} \newcommand{\R}{{\mathbb R}}
\newcommand{\N}{{\mathbb N}} \newcommand{\cF}{{\mathcal F}}
\newcommand{\gB}{{\mathfrak B}} \newcommand{\eps}{\epsilon}
\renewcommand\refname{参考文献} \def \qed {\hfill \vrule height6pt
  width 6pt depth 0pt} \topmargin -0.40in \oddsidemargin 0.08in
\evensidemargin 0.08in \marginparwidth0.00in \marginparsep 0.00in
\textwidth 15.5cm \textheight 24cm \pagestyle{myheadings}
\markboth{\rm \centerline{}} {\rm \centerline{}}
\begin{document}
\title{\huge{\bf{推论2.3.3.1}}} \author{\small{叶卢
    庆\footnote{叶卢庆(1992---),男,杭州师范大学理学院数学与应用数学专业
      本科在读,E-mail:h5411167@gmail.com}}\\{\small{杭州师范大学理学院,浙
      江~杭州~310036}}} \date{}
\maketitle

% ----------------------------------------------------------------------------------------
% ABSTRACT AND KEYWORDS
% ----------------------------------------------------------------------------------------




\vspace{30pt} % Some vertical space between the abstract and first section

% ----------------------------------------------------------------------------------------
% ESSAY BODY
% ----------------------------------------------------------------------------------------
\begin{theorem}
  若多项式 $f(x)$ 和 $g(x)$ 都与多项式 $h(x)$ 互素,那么乘积 $f(x)g(x)$
  也与 $h(x)$ 互素.
\end{theorem}
\begin{proof}[\textbf{证明}]
由于 $f(x)$ 与 $h(x)$ 互素,因此存在多项式 $u_1(x),v_1(x)$,使得
$$
u_1(x)f(x)+v_1(x)h(x)=1.
$$
由于 $g(x)$ 与 $h(x)$ 互素,因此存在多项式 $u_2(x),v_2(x)$,使得
$$
u_2(x)g(x)+v_2(x)h(x)=1.
$$
将如上两式相乘,可得
$$
u_1(x)u_2(x)f(x)g(x)+h(x)\Delta =1,
$$
其中 $\Delta $ 是一个多项式.因此 $f(x)g(x)$ 与 $h(x)$ 互素.  
\end{proof}
% BIBLIOGRAPHY
% ----------------------------------------------------------------------------------------
% 
\begin{thebibliography}{}
  % \bibitem[1]{tao}Terence Tao.陶哲轩实分析[M].王昆扬,译.北京:人民邮电出版
  %   社,2008:376
\end{thebibliography}
% ----------------------------------------------------------------------------------------
\end{document}








