\documentclass[twoside,11pt]{article} 
\usepackage{amsmath,amsfonts,bm}
\usepackage{hyperref}
\usepackage{amsthm} 
\usepackage{amssymb}
\usepackage{framed,mdframed}
\usepackage{graphicx,color} 
\usepackage{mathrsfs,xcolor} 
\usepackage[all]{xy}
\usepackage{fancybox} 
% \usepackage{CJKutf8}
\usepackage{xeCJK}
\newtheorem{theorem}{定理}
\newtheorem{lemma}{引理}
\newtheorem{corollary}{推论}
\newtheorem*{exercise}{习题}
\newtheorem*{example}{例}
\setCJKmainfont[BoldFont=Adobe Heiti Std R]{Adobe Song Std L}
% \usepackage{latexdef}
\def\ZZ{\mathbb{Z}} \topmargin -0.40in \oddsidemargin 0.08in
\evensidemargin 0.08in \marginparwidth 0.00in \marginparsep 0.00in
\textwidth 16cm \textheight 24cm \newcommand{\D}{\displaystyle}
\newcommand{\ds}{\displaystyle} \renewcommand{\ni}{\noindent}
\newcommand{\pa}{\partial} \newcommand{\Om}{\Omega}
\newcommand{\om}{\omega} \newcommand{\sik}{\sum_{i=1}^k}
\newcommand{\vov}{\Vert\omega\Vert} \newcommand{\Umy}{U_{\mu_i,y^i}}
\newcommand{\lamns}{\lambda_n^{^{\scriptstyle\sigma}}}
\newcommand{\chiomn}{\chi_{_{\Omega_n}}}
\newcommand{\ullim}{\underline{\lim}} \newcommand{\bsy}{\boldsymbol}
\newcommand{\mvb}{\mathversion{bold}} \newcommand{\la}{\lambda}
\newcommand{\La}{\Lambda} \newcommand{\va}{\varepsilon}
\newcommand{\be}{\beta} \newcommand{\al}{\alpha}
\newcommand{\dis}{\displaystyle} \newcommand{\R}{{\mathbb R}}
\newcommand{\N}{{\mathbb N}} \newcommand{\cF}{{\mathcal F}}
\newcommand{\gB}{{\mathfrak B}} \newcommand{\eps}{\epsilon}
\renewcommand\refname{参考文献} \def \qed {\hfill \vrule height6pt
  width 6pt depth 0pt} \topmargin -0.40in \oddsidemargin 0.08in
\evensidemargin 0.08in \marginparwidth0.00in \marginparsep 0.00in
\textwidth 15.5cm \textheight 24cm \pagestyle{myheadings}
\markboth{\rm \centerline{}} {\rm \centerline{}}
\begin{document}
\title{\huge{\bf{习题2.2.6}}} \author{\small{叶卢
    庆\footnote{叶卢庆(1992---),男,杭州师范大学理学院数学与应用数学专业
      本科在读,E-mail:h5411167@gmail.com}}\\{\small{杭州师范大学理学院,浙
      江~杭州~310036}}} \date{}
\maketitle

% ----------------------------------------------------------------------------------------
% ABSTRACT AND KEYWORDS
% ----------------------------------------------------------------------------------------




\vspace{30pt} % Some vertical space between the abstract and first section
\begin{exercise}[2.2.6]
考虑有理数域上多项式
$$
f_{n,k}(x)=(x+1)^{k+n}+(2x)(x+1)^{k+n-1}+\cdots+(2x)^k(x+1)^n,
$$
这里 $k,n$ 都是非负整数.证明
$$
x^{k+1}|(x-1)f_{n,k}(x)+(x+1)^{k+n+1}.
$$
\end{exercise}
\begin{proof}[\textbf{证明}]
采用数学归纳法.当 $k=0$ 时,我们先证明
$$
x|(x-1) (x+1)^n+(x+1)^{n+1}.
$$
也就是证明
$$
x|(x+1)^n(x-1+x+1).
$$
成立.假设 $k=p$ 的时候,命题成立.
则 $k=p+1$ 时,我们来看
\begin{align*}
  (x+1)^{p+1+n}+(2x)(x+1)^{p+n}+\cdots+(2x)^{p+1}(x+1)^n&=(x+1)^{p+1+n}+2xf_{n,p}(x)
\end{align*}
因此,
\begin{align*}
  (x-1)f_{n,p+1}(x)+(x+1)^{p+n+2}&=(x-1)(x+1)^{p+n+1}+2x(x-1)f_{n,p}(x)+(x+1)^{p+n+2}\\&=2x((x+1)^{p+n+1}+(x-1)f_{n,p}(x)).
\end{align*}
结合假设,可得
$$
x^{p+2}|2x((x+1)^{p+n+1}+(x-1)f_{n,p}(x)).
$$
因此根据数学归纳法,命题成立.
\end{proof}
% ----------------------------------------------------------------------------------------
% ESSAY BODY
% ----------------------------------------------------------------------------------------

% BIBLIOGRAPHY
% ----------------------------------------------------------------------------------------
% 
% ----------------------------------------------------------------------------------------
\end{document}








