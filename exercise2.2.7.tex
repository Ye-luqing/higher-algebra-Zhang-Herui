\documentclass[twoside,11pt]{article} 
\usepackage{amsmath,amsfonts,bm}
\usepackage{hyperref}
\usepackage{amsthm} 
\usepackage{amssymb}
\usepackage{framed,mdframed}
\usepackage{graphicx,color} 
\usepackage{mathrsfs,xcolor} 
\usepackage[all]{xy}
\usepackage{fancybox} 
% \usepackage{CJKutf8}
\usepackage{xeCJK}
\newtheorem{theorem}{定理}
\newtheorem{lemma}{引理}
\newtheorem{remark}{注}
\newtheorem{corollary}{推论}
\newtheorem*{exercise}{习题}
\newtheorem*{example}{例}
\setCJKmainfont[BoldFont=Adobe Heiti Std R]{Adobe Song Std L}
% \usepackage{latexdef}
\def\ZZ{\mathbb{Z}} \topmargin -0.40in \oddsidemargin 0.08in
\evensidemargin 0.08in \marginparwidth 0.00in \marginparsep 0.00in
\textwidth 16cm \textheight 24cm \newcommand{\D}{\displaystyle}
\newcommand{\ds}{\displaystyle} \renewcommand{\ni}{\noindent}
\newcommand{\pa}{\partial} \newcommand{\Om}{\Omega}
\newcommand{\om}{\omega} \newcommand{\sik}{\sum_{i=1}^k}
\newcommand{\vov}{\Vert\omega\Vert} \newcommand{\Umy}{U_{\mu_i,y^i}}
\newcommand{\lamns}{\lambda_n^{^{\scriptstyle\sigma}}}
\newcommand{\chiomn}{\chi_{_{\Omega_n}}}
\newcommand{\ullim}{\underline{\lim}} \newcommand{\bsy}{\boldsymbol}
\newcommand{\mvb}{\mathversion{bold}} \newcommand{\la}{\lambda}
\newcommand{\La}{\Lambda} \newcommand{\va}{\varepsilon}
\newcommand{\be}{\beta} \newcommand{\al}{\alpha}
\newcommand{\dis}{\displaystyle} \newcommand{\R}{{\mathbb R}}
\newcommand{\N}{{\mathbb N}} \newcommand{\cF}{{\mathcal F}}
\newcommand{\gB}{{\mathfrak B}} \newcommand{\eps}{\epsilon}
\renewcommand\refname{参考文献} \def \qed {\hfill \vrule height6pt
  width 6pt depth 0pt} \topmargin -0.40in \oddsidemargin 0.08in
\evensidemargin 0.08in \marginparwidth0.00in \marginparsep 0.00in
\textwidth 15.5cm \textheight 24cm \pagestyle{myheadings}
\markboth{\rm \centerline{}} {\rm \centerline{}}
\begin{document}
\title{\huge{\bf{习题2.2.7}}} \author{\small{叶卢
    庆\footnote{叶卢庆(1992---),男,杭州师范大学理学院数学与应用数学专业
      本科在读,E-mail:h5411167@gmail.com}}\\{\small{杭州师范大学理学院,浙
      江~杭州~310036}}} \date{}
\maketitle

% ----------------------------------------------------------------------------------------
% ABSTRACT AND KEYWORDS
% ----------------------------------------------------------------------------------------


\vspace{30pt} % Some vertical space between the abstract and first section

% ----------------------------------------------------------------------------------------
% ESSAY BODY
% ----------------------------------------------------------------------------------------
\begin{exercise}[2.2.7]
  证明 $x^d-1$ 整除 $x^n-1$ 当且仅当 $d$ 整除 $n$.
\end{exercise}
\begin{proof}[\textbf{证明}]
$\Leftarrow:$此时,不妨设 $n=kd$,此时,易得
$$
x^n-1=(x^{d})^k-1^k=(x^d-1)\Delta.
$$
其中 $\Delta $ 是一个多项式.因此 $x^d-1|x^n-1$.\\

$\Rightarrow:$ $x^d-1|x^n-1$ 说明 $n\geq d$.假若 $d\not|n$,则根据带余
除法,存在唯一的 $q,r$,使得
$$
n=qd+r,
$$
其中 $0< r<d$.则
$$
x^n-1=x^{qd+r}-1.
$$
而
$$
x^{qd+r}-1=x^{qd}x^r-x^r+x^r-1=x^r(x^{qd}-1)+(x^r-1)=(x^d-1)x^r\Delta'+(x^r-1).
$$
易得 $x^d-1\not|x^r-1$,因此,$x^d-1\not |x^n-1$,矛盾.因此 $d|n$.
\end{proof}
\begin{remark}
  这个结论建立了 $d|n$ 和 $x^d-1|x^n-1$ 的对应关系,个人认为会有大用.
\end{remark}
% BIBLIOGRAPHY
% ----------------------------------------------------------------------------------------
% 
% ----------------------------------------------------------------------------------------
\end{document}








